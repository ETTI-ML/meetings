\documentclass[ignorenonframetext,]{beamer}
\setbeamertemplate{caption}[numbered]
\setbeamertemplate{caption label separator}{: }
\setbeamercolor{caption name}{fg=normal text.fg}
\beamertemplatenavigationsymbolsempty
\usepackage{lmodern}
\usepackage{amssymb,amsmath}
\usepackage{ifxetex,ifluatex}
\usepackage{fixltx2e} % provides \textsubscript
\ifnum 0\ifxetex 1\fi\ifluatex 1\fi=0 % if pdftex
  \usepackage[T1]{fontenc}
  \usepackage[utf8]{inputenc}
\else % if luatex or xelatex
  \ifxetex
    \usepackage{mathspec}
  \else
    \usepackage{fontspec}
  \fi
  \defaultfontfeatures{Ligatures=TeX,Scale=MatchLowercase}
\fi
% use upquote if available, for straight quotes in verbatim environments
\IfFileExists{upquote.sty}{\usepackage{upquote}}{}
% use microtype if available
\IfFileExists{microtype.sty}{%
\usepackage{microtype}
\UseMicrotypeSet[protrusion]{basicmath} % disable protrusion for tt fonts
}{}
\newif\ifbibliography
\hypersetup{
            pdftitle={Meeting 1},
            pdfborder={0 0 0},
            breaklinks=true}
\urlstyle{same}  % don't use monospace font for urls

% Prevent slide breaks in the middle of a paragraph:
\widowpenalties 1 10000
\raggedbottom

\AtBeginPart{
  \let\insertpartnumber\relax
  \let\partname\relax
  \frame{\partpage}
}
\AtBeginSection{
  \ifbibliography
  \else
    \let\insertsectionnumber\relax
    \let\sectionname\relax
    \frame{\sectionpage}
  \fi
}
\AtBeginSubsection{
  \let\insertsubsectionnumber\relax
  \let\subsectionname\relax
  \frame{\subsectionpage}
}

\setlength{\parindent}{0pt}
\setlength{\parskip}{6pt plus 2pt minus 1pt}
\setlength{\emergencystretch}{3em}  % prevent overfull lines
\providecommand{\tightlist}{%
  \setlength{\itemsep}{0pt}\setlength{\parskip}{0pt}}
\setcounter{secnumdepth}{0}
\usetheme{Madrid}
\setbeamertemplate{footline}{} % disable footer
%\usecolortheme{whale}
\setbeamertemplate{itemize items}[default]
\setbeamertemplate{enumerate items}[default]

\setbeamersize{text margin left=15pt}

\title{Meeting 1}
\subtitle{ETTI-ML Meetings}
\date{13.03.2018}

\begin{document}
\frame{\titlepage}

\begin{frame}{Topics}

\begin{enumerate}
\def\labelenumi{\arabic{enumi}.}
\item
  Location / resources
\item
  Review: ``Deep Learning for Computer Vision''

  \begin{itemize}
  \tightlist
  \item
    Starter Bundle
  \item
    Practitioner Bundle
  \end{itemize}
\item
  GPU on the cloud: floydhub.com
\item
  Review: ``Deep Image Prior''
\end{enumerate}

\end{frame}

\begin{frame}{Location / resources}

\begin{itemize}
\tightlist
\item
  Let's hold common stuff (presentations, paper etc.) in a common place
\end{itemize}

\begin{enumerate}
\def\labelenumi{\arabic{enumi}.}
\tightlist
\item
  Github: https://github.com/ETTI-ML

  \begin{itemize}
  \tightlist
  \item
    Organization: ETTI-ML
  \item
    Made a repository for meetings: https://github.com/ETTI-ML/meetings
  \item
    Only public repositories for free accounts
  \item
    Alternatives for private repos: gitlab.com, bitbucket.com
  \end{itemize}
\item
  Shared paper database: Zotero (zotero.com)

  \begin{itemize}
  \tightlist
  \item
    Keep a common database of papers, review notes, links etc.
  \item
    A little cumbersome
  \item
    Alternatives: Mendeley, EndNote, Paperpile etc
  \end{itemize}
\end{enumerate}

\end{frame}

\begin{frame}{Review: ``Deep Learning for Computer Vision''}

\begin{itemize}
\item
  Review: ``Deep Learning for Computer Vision, With Python'', Dr.~Adrian
  Rosebrock, 1st Ed.
\item
  Book + code examples + Virtual Machine
\item
  Comes in three flavours:

  \begin{itemize}
  \tightlist
  \item
    Starter Bundle
  \item
    Practitioner Bundle
  \item
    ImageNet Bundle (not available)
  \end{itemize}
\end{itemize}

\end{frame}

\begin{frame}{Review: ``Deep Learning for Computer Vision''}

\begin{itemize}
\item
  Topic: Deep CNNs for image classification
\item
  Style:

  \begin{itemize}
  \tightlist
  \item
    little theory
  \item
    examples in Python, explained step by step
  \end{itemize}
\end{itemize}

\end{frame}

\begin{frame}{Starter Bundle}

\begin{itemize}
\tightlist
\item
  Starter Bundle

  \begin{itemize}
  \tightlist
  \item
    the easiest, contains the basics
  \end{itemize}
\item
  Topics:

  \begin{itemize}
  \tightlist
  \item
    Basics

    \begin{itemize}
    \tightlist
    \item
      Image classification basics
    \item
      Basic datasets
    \item
      Stochastic Gradient Descent
    \end{itemize}
  \item
    Neural Network basic architectures

    \begin{itemize}
    \tightlist
    \item
      Basic layer types
    \item
      Backprop
    \item
      CNN building blocks
    \item
      Example: recognizing handwritten digits (MNIST) with LeNet
    \end{itemize}
  \item
    Some tips \& tricks:

    \begin{itemize}
    \tightlist
    \item
      spotting underfiting / overfiting
    \item
      checkpointing
    \item
      visualize architectures
    \end{itemize}
  \end{itemize}
\end{itemize}

\end{frame}

\begin{frame}{Practitioner Bundle}

\begin{itemize}
\tightlist
\item
  Practitioner Bundle

  \begin{itemize}
  \tightlist
  \item
    More advanced tips \& tricks, but still easy from theoretic p.o.v.
  \end{itemize}
\item
  Topics

  \begin{itemize}
  \tightlist
  \item
    Advanced (state-of-the-art) CNN architectures for image
    classification: VGG, GoogLeNet, ResNet\\
  \item
    Adaptation: train / replace only top layers, keep pre-trained lower
    layers
  \item
    Various alternatives to SGD (RMSprop etc)
  \item
    More handy tips \& tricks: data augmentation, preprocessing, work
    with HDF5 files
  \item
    Works with larger datasets (Kaggle, subset of ImageNet)
  \end{itemize}
\end{itemize}

\end{frame}

\begin{frame}{Side topics}

\begin{itemize}
\tightlist
\item
  OpenCV is really for image processing
\item
  This guy has a similar book for OpenCV: ``Practical Python and OpenCV
  + Case Studies''
\end{itemize}

\end{frame}

\begin{frame}{GPU on the cloud: floydhub.com}

\begin{itemize}
\tightlist
\item
  floydhub.com
\item
  transfer code automatically to their site \& run
\item
  works with Jupyter notebooks (maybe also plain .py files)
\item
  can be run / controlled from Linux command line (nice)
\item
  affordable: Standard GPU, Tesla K80 with 12GB Memory, preemptible: 7\$
  / 10h ()
\end{itemize}

\end{frame}

\end{document}
